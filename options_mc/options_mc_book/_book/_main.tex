% Options for packages loaded elsewhere
\PassOptionsToPackage{unicode}{hyperref}
\PassOptionsToPackage{hyphens}{url}
%
\documentclass[
]{book}
\usepackage{amsmath,amssymb}
\usepackage{lmodern}
\usepackage{iftex}
\ifPDFTeX
  \usepackage[T1]{fontenc}
  \usepackage[utf8]{inputenc}
  \usepackage{textcomp} % provide euro and other symbols
\else % if luatex or xetex
  \usepackage{unicode-math}
  \defaultfontfeatures{Scale=MatchLowercase}
  \defaultfontfeatures[\rmfamily]{Ligatures=TeX,Scale=1}
\fi
% Use upquote if available, for straight quotes in verbatim environments
\IfFileExists{upquote.sty}{\usepackage{upquote}}{}
\IfFileExists{microtype.sty}{% use microtype if available
  \usepackage[]{microtype}
  \UseMicrotypeSet[protrusion]{basicmath} % disable protrusion for tt fonts
}{}
\makeatletter
\@ifundefined{KOMAClassName}{% if non-KOMA class
  \IfFileExists{parskip.sty}{%
    \usepackage{parskip}
  }{% else
    \setlength{\parindent}{0pt}
    \setlength{\parskip}{6pt plus 2pt minus 1pt}}
}{% if KOMA class
  \KOMAoptions{parskip=half}}
\makeatother
\usepackage{xcolor}
\usepackage{longtable,booktabs,array}
\usepackage{calc} % for calculating minipage widths
% Correct order of tables after \paragraph or \subparagraph
\usepackage{etoolbox}
\makeatletter
\patchcmd\longtable{\par}{\if@noskipsec\mbox{}\fi\par}{}{}
\makeatother
% Allow footnotes in longtable head/foot
\IfFileExists{footnotehyper.sty}{\usepackage{footnotehyper}}{\usepackage{footnote}}
\makesavenoteenv{longtable}
\usepackage{graphicx}
\makeatletter
\def\maxwidth{\ifdim\Gin@nat@width>\linewidth\linewidth\else\Gin@nat@width\fi}
\def\maxheight{\ifdim\Gin@nat@height>\textheight\textheight\else\Gin@nat@height\fi}
\makeatother
% Scale images if necessary, so that they will not overflow the page
% margins by default, and it is still possible to overwrite the defaults
% using explicit options in \includegraphics[width, height, ...]{}
\setkeys{Gin}{width=\maxwidth,height=\maxheight,keepaspectratio}
% Set default figure placement to htbp
\makeatletter
\def\fps@figure{htbp}
\makeatother
\setlength{\emergencystretch}{3em} % prevent overfull lines
\providecommand{\tightlist}{%
  \setlength{\itemsep}{0pt}\setlength{\parskip}{0pt}}
\setcounter{secnumdepth}{5}
\usepackage{booktabs}
\ifLuaTeX
  \usepackage{selnolig}  % disable illegal ligatures
\fi
\usepackage[]{natbib}
\bibliographystyle{plainnat}
\IfFileExists{bookmark.sty}{\usepackage{bookmark}}{\usepackage{hyperref}}
\IfFileExists{xurl.sty}{\usepackage{xurl}}{} % add URL line breaks if available
\urlstyle{same} % disable monospaced font for URLs
\hypersetup{
  pdftitle={Buying Derivatives - A Monte Carlo Approach},
  pdfauthor={Jake Rozran},
  hidelinks,
  pdfcreator={LaTeX via pandoc}}

\title{Buying Derivatives - A Monte Carlo Approach}
\author{Jake Rozran}
\date{2023-04-06}

\begin{document}
\maketitle

{
\setcounter{tocdepth}{1}
\tableofcontents
}
\hypertarget{introduction}{%
\chapter{Introduction}\label{introduction}}

Over the course of the last year, I have developed an algorithm that calculates
the expected outcome for security\footnote{In finance, a security refers to
  a tradable financial asset that has monetary value and can be bought or sold in
  a financial market. Securities include a wide range of assets, such as stocks,
  bonds, options, futures, and exchange-traded funds (ETFs), among others.}
derivatives\footnote{A derivative is a financial instrument whose value is derived from
  the value of an underlying security, such as a stock, bond, commodity, or currency.
  In other words, the value of a derivative depends on the value of the
  underlying asset. There are many types of derivatives, including options,
  futures, forwards, and swaps.For example, an option is a type of
  derivative that gives the holder the right, but not the obligation, to buy or
  sell an underlying asset at a specific price within a specified time period.}.
In this paper, I am specifically
referring to put options\footnote{A put option is a type of financial contract that
  gives the holder the right, but not the obligation, to \emph{sell} an underlying
  security at a predetermined price (known as the strike price) within a specific
  period of time (known as the strike date).In other words, a put option
  gives the holder the ability to sell an asset at a certain price, even if the market
  price of the asset falls below the strike price. This can be useful for
  investors who believe that the price of an asset is likely to decline, as it
  allows them to profit from the decline without actually owning the asset.
  For example, if an investor buys a put option for a certain stock with a strike
  price of \$50 and an expiration date of one month, and the stock's market price
  falls to \$40 during that time, the investor can exercise the put option to sell
  the stock at the higher strike price of \$50, making a profit of \$10 per share.
  However, if the stock's market price remains above the strike price of
  \$50, the investor may choose not to exercise the put option and will lose the
  premium paid to purchase the option.} and call options\footnote{A call option is a type
  of financial contract that gives the holder the right, but not the obligation,
  to \emph{buy} an underlying asset, such as a stock, bond, or commodity, at a
  predetermined price (known as the strike price) within a specific period of
  time.In other words, a call option gives the holder the ability to
  purchase an asset at a certain price, even if the market price of the asset
  rises above the strike price. This can be useful for investors who believe that
  the price of an asset is likely to increase, as it allows them to profit from
  the increase without actually owning the asset.For example, if an
  investor buys a call option for a certain stock with a strike price of \$50 and
  an expiration date of one month, and the stock's market price rises to \$60
  during that time, the investor can exercise the call option to buy the stock at
  the lower strike price of \$50, making a profit of \$10 per share.
  However, if the stock's market price remains below the strike price of \$50, the
  investor may choose not to exercise the call option and will lose the premium
  paid to purchase the option.} when I reference derivatives. To do this, I
calculate several probabilities\footnote{Probability is a measure of the likelihood or
  chance that a specific event will occur, expressed as a number between 0 and 1.
  A probability of 0 means that the event is impossible, while a
  probability of 1 means that the event is certain to occur. A probability of
  0.5 (or 50\%) means that the event has an equal chance of occurring or not
  occurring.} and expected values, then combine them to
understand if it is worthwhile to purchase a security derivative.

These probabilities are \protect\hyperlink{link-to-bootstrap}{bootstrapped} through the use of a
\protect\hyperlink{link-to-mc}{Monte Carlo simulation}. I am first simulating the price of the
security on the strike date. I am then able, with this information, to obtain
the probability the security will be above or below a certain strike and the
expected value of the security if it does go above (below) the strike for a call
(put). Combining this with the cost to purchase that derivative, I am able to
create an expected value for a range of security derivatives.

\hypertarget{ev}{%
\chapter{Calculating Expected Value}\label{ev}}

The expected value for each security derivative can be calculated as follows:

\begin{equation} 
E_p(k) = (P_{ae} * C) + (P_{be} * G)
\label{eq:puts}
\end{equation}

\begin{equation} 
E_c(k) = (P_{be} * C) + (P_{ae} * G)
\label{eq:calls}
\end{equation}

where

\begin{itemize}
\tightlist
\item
  \(E_p(k)\) is the expected value of a put at strike price \(k\)
\item
  \(E_c(k)\) is the expected value of a call at strike price \(k\)
\item
  \(P_{ae}\) is the probability a security price on the strike date is above the
  \emph{expected price}
\item
  \(P_{be}\) is the probability a security price on the strike date is below the
  \emph{expected price}
\item
  \(C\) is the cost to but the derivative
\item
  \(G\) is the expected gain if the derivative is in the money
\end{itemize}

\end{document}
